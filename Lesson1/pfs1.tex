% Python for Scientists: Lesson 1
% Author: Marshall Ward
%=============================================================================%
\documentclass[red]{beamer}

% Font configuration
\usepackage{pxfonts}
%\usepackage{eulervm}
%\usepackage{type1cm}
%\usepackage{fourier}

% Packages
\usepackage{listings}

\usepackage{color}
\definecolor{ltblue}{rgb}{0.9, 0.9, 1.0}
\definecolor{dkgreen}{rgb}{0,0.6,0}
\definecolor{gray}{rgb}{0.5,0.5,0.5}
\definecolor{mauve}{rgb}{0.58,0,0.82}

% Beamer configuration
\usetheme{Frankfurt}
%-----------------------------------------------------------------------------%
\AtBeginSection[]
{
   \begin{frame}
        \frametitle{Outline}
        \tableofcontents[currentsection]
    \end{frame}
}
%-----------------------------------------------------------------------------%
\title{Python for Scientists: Basics}
%=============================================================================%
\begin{document}
% Listings settings
\lstset{
    language=Python,
    basicstyle=\ttfamily,
    showspaces=false,
    showstringspaces=false,
    backgroundcolor=\color{ltblue},
    numberstyle=\tiny\color{gray},        % line number style
    keywordstyle=\color{blue},          % keyword style
    commentstyle=\color{dkgreen},       % comment style
    stringstyle=\color{mauve}         % string literal style
}
%-----------------------------------------------------------------------------%
\begin{frame}
    \titlepage
\end{frame}
%=============================================================================%
\section[Basics]{Basic Python}
%-----------------------------------------------------------------------------%
\frame{
    \frametitle{Starting up Python}

    Open a terminal and type the following:
    
    \begin{block}{}
    \lstinline|$ python|
    \end{block}

    This is the \textit{interactive shell}.
    
    \begin{itemize}
        \item What version of Python are you running?
        \item Quit the shell by typing Control-D (EOF)
    \end{itemize}
}
%-----------------------------------------------------------------------------%
\begin{frame}[fragile]
    \frametitle{Integers and Floating Point Numbers}
    
    Python distinguishes between integers and floating point:
    \begin{itemize}
        \item \lstinline|>>> 1 + 2 - 3 * 4 / 5|\\
            \onslide<2->{\lstinline|1|}
        \item \lstinline|>>> 1.+ 2.* 3./ 4.- 5.| \\
            \onslide<2->{\lstinline|0.6000000000000001|}
    \end{itemize}
    
    \onslide<2->{Watch your decimal points!}
    \\~\\
    \onslide<2->{\small Note: \lstinline|4 / 5 = 0.8| in Python 3.x}
\end{frame}
%-----------------------------------------------------------------------------%
\begin{frame}[fragile]
    \frametitle{New Syntax}
    
    \begin{itemize}
        \item Exponents: \lstinline|>>> 3**4| \\
            \onslide<2->{\lstinline|81|}
        \item Complex numbers: \lstinline|>>> (1+2j)*(2+3J)| \\
            \onslide<2->{\lstinline|(-4+7j)|}
        \item Last output: \lstinline|>>> _ + 1| \\
            \onslide<2->{\lstinline|(-3+7j)|}
        \item Truncated division: \lstinline|7.// 4.| \\
            \onslide<2->{\lstinline|1.0|}
    \end{itemize}
\end{frame}
%----------------------------------------------------------------------------%
\begin{frame}[fragile]
    \frametitle{Logical Operators}

    Python has the usual logical operators:
    \begin{itemize}
        \item \lstinline|>>> 1 < 2|\\
            \onslide<2->{\lstinline|True|}
        \item \lstinline|>>> 3 >= 2|\\
            \onslide<2->{\lstinline|False|}
        \item \lstinline|>>> 5 > 4 and 2 == 3|\\
            \onslide<2->{\lstinline|False|}
        \item \lstinline|>>> 1 < 2 < 3|\\
            \onslide<2->{\lstinline|True|}
    \end{itemize}
    
    \begin{block}
    \onslide<2->{
        \lstinline|1 < 2 < 3| is shorthand: \lstinline|1 < 2 and 2 < 3|
        \\~\\
        Try \lstinline|1 < 2 == 2| and \lstinline|(1 < 2) == 2|.
    }
    \end{block}
\end{frame}
%----------------------------------------------------------------------------%
\frame{    
    \frametitle{Modular Arithmetic}
    
    Modular arithmetic is not quite like C and Fortran:
    \begin{columns}
        \column{0.5\textwidth}
        \begin{itemize}
            \item \lstinline|>>> 7 // 4|\\
                \onslide<2->{\lstinline|1|}
            \item \lstinline|>>> 7 \% 4|\\
                \onslide<2->{\lstinline|3|}
        \end{itemize}
        \column{0.5\textwidth}
        \begin{itemize}
            \item \lstinline|>>> -7 // 4|\\
                \onslide<2->{\lstinline|-2|}
            \item \lstinline|>>> -7 \% 4|\\
                \onslide<2->{\lstinline|1|}
        \end{itemize}
    \end{columns}
    
    \vspace{10pt}
    \onslide<3>{
        \begin{block}{}
        For \lstinline|q = x / N| and \lstinline|r = x \% N|, Python solves
            \[ x = q N + r\]
        so that $0 \leq r < N$ for $N>0$.
        \end{block}
    }
}
%-----------------------------------------------------------------------------%
\begin{frame}[fragile]
    \frametitle{Variables}
    
    Create variables at any time (\textit{dynamic typing}):
    \begin{lstlisting}
x = 123.
y = 456.
z = x + y
x = 'Now x is a string.'
z = x + y # ???
    \end{lstlisting}

    Assigning several variables:
    \begin{lstlisting}
x = y = z = 1
    \end{lstlisting}

\end{frame}
%-----------------------------------------------------------------------------%
\begin{frame}[fragile]
    \frametitle{\texttt{math} and Modules}
    
    \textit{Modules} let you import external functions, variables, etc.
    \\~\\
    \texttt{math} provides standard mathematics functions.

    Import the \texttt{math} module and try some functions:
    \begin{lstlisting}
import math
x = math.pi
y = math.sin(x/2.)
    \end{lstlisting}

    (Note: \texttt{numpy} provides most of the same functions.)
\end{frame}
%-----------------------------------------------------------------------------%
\begin{frame}[fragile]
    \frametitle{Import syntax}

    You can rename module imports
    \begin{lstlisting}
import math as m
x = m.pi
y = m.sin(x/2.)
    \end{lstlisting}

    or import specific functions:
    \begin{lstlisting}
from math import pi
    \end{lstlisting}

    or the entire contents (not recommended though):
    \begin{lstlisting}
from math import *
    \end{lstlisting}
\end{frame}
%-----------------------------------------------------------------------------%
\begin{frame}[fragile]
    \frametitle{Basic I/O}
    
    Basic statement printing ('echo'):
    \begin{lstlisting}
>>> print 'Hello Python.'
Hello Python.
    \end{lstlisting}

    Commas separate quantities by spaces:
    \begin{lstlisting}
>>> x = 1
>>> y = 2
>>> print 'x =', x, 'and y =', y
x = 1 and y = 2
    \end{lstlisting}
\end{frame}
%=============================================================================%
\section[Scripts]{Scripting}
%-----------------------------------------------------------------------------%
\begin{frame}[fragile]
    \frametitle{Running Python Scripts}

    Create a file named \texttt{hello.py} containing the following:

    \begin{lstlisting}
print "This is my first Python script."
name = raw_input("Type your name: ")
print "Hello", name
print "Goodbye."
    \end{lstlisting}

    Now run the script by typing the following:
    \begin{block}{}
        \texttt{\$ python hello.py}
    \end{block}
\end{frame}
%-----------------------------------------------------------------------------%
\begin{frame}[fragile]
    \frametitle{Self-executing scripts}

    Put the following as your first line in the script:
    \begin{lstlisting}
#!/usr/bin/env python
    \end{lstlisting}

    Then make the script executable by typing in the terminal:
    \begin{block}{}
        \texttt{\$ chmod +x hello.py}
    \end{block}

    Now you can run the script as an executable:
    \begin{block}{}
        \texttt{./hello.py}
    \end{block}
\end{frame}
%=============================================================================%
\section[Lists]{Lists and Indexing}
%-----------------------------------------------------------------------------%
\begin{frame}[fragile]
    \frametitle{Lists}
    
    Lists are sequences of data
    \begin{lstlisting}
x = [0, 1, 2, 2]
y = [4, 4.0, 'four']
z = [x, y, 1, [2,3] ]
    \end{lstlisting}
    Lists can contain anything, even other lists.

\end{frame}
%-----------------------------------------------------------------------------%
\begin{frame}[fragile]
    \frametitle{List Indexing}

    Indexing begins at zero, denoted by \texttt{[]}:
    \begin{lstlisting}
>>> x = [1,2,3,4,5]
>>> x[0]
1
>>> x[4]
5
    \end{lstlisting}

    \lstinline!range(N)! will generate an 'index list' of size $N$:
    \begin{lstlisting}
range(10)
[0, 1, 2, 3, 4, 5, 6, 7, 8, 9]
    \end{lstlisting}
\end{frame}
%-----------------------------------------------------------------------------%
\begin{frame}
    \frametitle{Slicing}

    \texttt{x[m:n]} outputs a segment from $m \leq i < n$.
    \\~\\    
    For \texttt{x = range(8)}, try the following:
    \begin{itemize}
        \item \texttt{x[3:6]}\\
            \onslide<2->{\texttt{[3, 4, 5]}}
        \item \texttt{x[:6]}\\
            \onslide<2->{\texttt{[0, 1, 2, 3, 4, 5]}}
        \item \texttt{x[3:]}\\
            \onslide<2->{\texttt{[3, 4, 5, 6, 7]}}
        \item \texttt{x[:]}\\
            \onslide<2->{\texttt{[0, 1, 2, 3, 4, 5, 6, 7]}}
    \end{itemize}
    When omitted, \texttt{m = 0} and \texttt{n = len(x)}
\end{frame}
%-----------------------------------------------------------------------------%
\begin{frame}[fragile]
    \frametitle{Striding}

    A second colon denotes stride, e.g. \lstinline|x[m:n:k]|

    \lstinline|x[::3]| returns every third element

    \begin{lstlisting}
>>> x = range(10)
>>> x[::2]
[0, 2, 4, 6, 8]
>>> x[2::3]
[2, 5, 8]
>>> x[::-1]     # Reverse stride
[9, 8, 7, 6, 5, 4, 3, 2, 1, 0]
    \end{lstlisting}
\end{frame}
%-----------------------------------------------------------------------------%
\begin{frame}[fragile]
    \frametitle{Inverse index}

    Python supports \textit{negative indexing}:
    \\~\\
    \lstinline|x = range(8)|
    \begin{itemize}
        \item \lstinline|x[-1]| \\
            \onslide<2->{\lstinline|[7]|}
        \item \lstinline|x[:-3]| \\
            \onslide<2->{\lstinline|[0, 1, 2, 3, 4]|}
        \item \lstinline|x[2:-2]| \\
            \onslide<2->{\lstinline|[2, 3, 4, 5]|}
    \end{itemize}

    \begin{block}{}
        \lstinline|x[-k]| is just shorthand for \lstinline|x[N-k]|.
    \end{block}
\end{frame}
%-----------------------------------------------------------------------------%
\begin{frame}[fragile]
    \frametitle{Index Quiz}

    Write a script that reads a number $N$ using this command:
    \lstinline|N = int(raw_input('Enter N: '))|
    \\~\\
    Create a list using \lstinline|range(N)| and print the following:
    \begin{itemize}
        \item The entire list
        \item The reversed list
        \item The first six elements
        \item The last three
        \item Everything but the last three
        \item Every third element between the first four and last four
        \item The first half of the list
        \item the last five elements, reversed
    \end{itemize}
\end{frame}
%-----------------------------------------------------------------------------%
\frame{
    \frametitle{Index Solution}

    One solution, at least.
    \lstinputlisting{scripts/index_quiz.py}
}
%-----------------------------------------------------------------------------%
\begin{frame}[fragile]
    \frametitle{Tuples}

    Tuples are like lists, except they cannot be modified:
    \begin{lstlisting}
>>> x = tuple([1,2,3])
>>> x
(1, 2, 3)
>>> x[0] = 5
    \end{lstlisting}
    A favorite swap trick:
    \begin{lstlisting}
>>> x = 1
>>> y = 2
>>> x, y = y, x
    \end{lstlisting}
    Parentheses are often implicit.
\end{frame}
%-----------------------------------------------------------------------------%
\begin{frame}[fragile]
    \frametitle{Dicts}

    Dicts are mappings:
    \begin{lstlisting}
>>> d = {'one': 1, 'two': 2, 'three': 3}
>>> d['one']
1
    \end{lstlisting}
    Like lists, with generic indexing
    \\~\\
    Fun way to create a dict:
    \begin{lstlisting}
>>> x = ['one', 'two', 'three']
>>> y = [1, 2, 3]
>>> zip(x, y)
[('one', 1), ('two', 2), ('three', 3)]
>>> d = dict( zip(x,y) )
    \end{lstlisting}
\end{frame}
%=============================================================================%
\section[Flow Control]{Basic Flow Control}
%-----------------------------------------------------------------------------%
\begin{frame}[fragile]
    \frametitle{\texttt{if-elif-else} control}

    Python flow control is minimal:
    \begin{lstlisting}
x = 1
if x > 0:
    print "x is positive"
elif x < 0:
    print "x is negative"
else:
    print "x is zero"
    \end{lstlisting}

    Notice:
    \begin{itemize}
        \item Conditionals end with \lstinline|:|
        \item There is no \texttt{endif}, closing bracket, etc.
        \item Indentation is \textit{mandatory}!
    \end{itemize}
\end{frame}
%-----------------------------------------------------------------------------%
\begin{frame}[fragile]
    \frametitle{\texttt{while} loops}
    
    Python has a \lstinline|while| loop:
    \begin{lstlisting}
x = 0
while x < 10:
    print x
    x += 1
    \end{lstlisting}
    but in practice they are rarely used.
\end{frame}
%-----------------------------------------------------------------------------%
\begin{frame}[fragile]
    \frametitle{\texttt{for} loops}
    
    The most common Python loop:
    \begin{lstlisting}
for i in range(10):
    print i
    \end{lstlisting}

    In Python, you \textit{iterate} through the list:
    \begin{lstlisting}
x = [1, 3, 7, 'apple', 'banana']
for i in x:
    print i
    \end{lstlisting}
    Iterating through elements is preferred (and usually fastest).

\end{frame}
%-----------------------------------------------------------------------------%
\begin{frame}[fragile]
    \frametitle{List Comprehensions}

    \textit{List comprehensions} are rapid methods for list generation:
    \\~\\
    Create a list of even numbers:
    \begin{lstlisting}
x = [i for i in range(10) if i%2 == 0]
    \end{lstlisting}

    Create a list of squared numbers:
    \begin{lstlisting}
y = [i**2 for i in range(10)]
    \end{lstlisting}
\end{frame}
%-----------------------------------------------------------------------------%
\begin{frame}[fragile]
    \frametitle{Functions}

    Minimal quadratic function:
    \begin{lstlisting}
def square(x):
    return x*x
    \end{lstlisting}

    Anything can be a function argument, even other functions:
    \begin{lstlisting}
def f(x):
    return x**2

def g(f, N):
    return [f(i) for i in range(N)]
    \end{lstlisting}
    Try \lstinline|f(4)| and \lstinline|g(f, 10)|.

\end{frame}
%-----------------------------------------------------------------------------%
\begin{frame}[fragile]
    \frametitle{Functions in scripts}

    Try this script:
    \begin{lstlisting}
#!/usr/bin/env python

x = 12.
print square(x)

def square(y):
    return y**2
    \end{lstlisting}
    Why doesn't it work?
\end{frame}
%-----------------------------------------------------------------------------%
\begin{frame}[fragile]
    \frametitle{Using \lstinline|__main__| in scripts}

    The previous script defined \lstinline|square| \textit{after} using it!

    Use \lstinline|__main__| to control script initiation:
    \begin{lstlisting}
#!/usr/bin/env python

def main():
    x = 12.
    print square(x)

def square(y):
    return y**2

if __name__ == '__main__':
    main()
    \end{lstlisting}
    What if the \lstinline|__name__| statement were first?
\end{frame}
%=============================================================================%
\section[Objects]{Objects in Python}
%-----------------------------------------------------------------------------%
\begin{frame}[fragile]
    \frametitle{Everthing is an object}

    In Python, everything is an \emph{object}: \\
        integers, strings, lists, functions, etc.

    All objects have a \lstinline|type()|:
    \begin{lstlisting}
>>> type(3)
>>> type([1,2,3])
>>> type('Hello')
>>> type(type)
    \end{lstlisting}

    Variables are just \emph{references} to the objects in memory:
    \begin{lstlisting}
>>> x = 3
>>> type(x)
>>> x is 3
    \end{lstlisting}
\end{frame}
%-----------------------------------------------------------------------------%
\begin{frame}[fragile]
    \frametitle{Object Methods}

    \lstinline|dir(x)| returns all the contents (i.e. objects) of \lstinline|x|.

    Create a list and look inside:
    \begin{lstlisting}
>>> x = range(4)
>>> dir(x)
    \end{lstlisting}

    The dot (.) lets you access these contents of an object, e.g.
    \begin{lstlisting}
>>> x.append(9)
>>> x
[0, 1, 2, 3, 4, 9]
    \end{lstlisting}

    Use \lstinline|help()| on some of the methods and try them out.\\
    (Ignore the ones with underscores)

\end{frame}
%-----------------------------------------------------------------------------%
\begin{frame}[fragile]
    \frametitle{Variables and References}

    There are \textit{immutable} and \textit{mutable} objects.

    Immutable objects cannot be changed: numbers, strings, tuples
    \begin{lstlisting}
>>> x = 3   # x points to 3
>>> y = x   # y points to what x points to
>>> x = 4   # Now x points to 4
>>> x
4
>>> y
3
    \end{lstlisting}
    The number 3 did not (and cannot) change, so \lstinline|y| still points to 3.
    \\~\\
    Do not think that \lstinline|y| 'contains' the value of 3!
\end{frame}
%-----------------------------------------------------------------------------%
\begin{frame}[fragile]
    \frametitle{Variables and References}

    Now let \lstinline|x| point to a \textit{mutable} list:
    \begin{lstlisting}
>>> x = [1,2,3] # [1,2,3] is created
                # and x points to it
>>> y = x       # Now y points to the list
>>> x[0] = 7    # Change part of x
>>> y           # What is y?
    \end{lstlisting}
    
    NumPy arrays are mutable, and behave similarly.
\end{frame}
%=============================================================================%
\section[Strings]{String Manipulation}
%-----------------------------------------------------------------------------%
\begin{frame}[fragile]
    \frametitle{String Manipulation}

    Strings support a \lstinline|printf|-like syntax:
    \begin{lstlisting}
>>> x = 1
>>> print 'x = %i' % x
>>> y = 2.0
>>> print 'x = %i and y = %.2f.' % (x,y)
    \end{lstlisting}

    There are also several methods to parse and manipulate strings.
\end{frame}
%=============================================================================%
\section[\lstinline|os|]{\texttt{os}: Interface to the OS}
%-----------------------------------------------------------------------------%
\frame{
    \frametitle{\lstinline|os|: Interface to the Operating System}

    \lstinline|import os| provides OS-level commands, e.g.:
    \begin{itemize}
        \item \lstinline|os.system(cmd)|\\
            Run \lstinline|cmd| in the shell (but see \lstinline|subprocess|)
        \item \lstinline|os.getcwd()|\\
            Return your current directory
        \item \lstinline|os.listdir(path)|\\
            List of all filenames in \lstinline|path|
        \item \lstinline|os.mkdir(path)|\\
            Create a directory named \lstinline|path|
        \item \lstinline|os.remove(path)|\\
            Delete the file at \lstinline|path|
        \item \lstinline|os.path.join(path1, path2)|\\
            Append \lstinline|path2| to \lstinline|path1| as an absolute path
    \end{itemize}
    Also: \lstinline|sys|, \lstinline|shutil|, \lstinline|subprocess|
}
%=============================================================================%
\end{document}
