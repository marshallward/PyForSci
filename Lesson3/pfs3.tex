% Python for Scientists: Lesson 3
% Author: Marshall Ward
%=============================================================================%
\documentclass[red]{beamer}

% Font configuration
\usepackage{pxfonts}
%\usepackage{eulervm}
%\usepackage{type1cm}
%\usepackage{fourier}

% Packages
\usepackage{listings}

\usepackage{color}
\definecolor{ltblue}{rgb}{0.9, 0.9, 1.0}
\definecolor{dkgreen}{rgb}{0,0.6,0}
\definecolor{gray}{rgb}{0.5,0.5,0.5}
\definecolor{mauve}{rgb}{0.58,0,0.82}

\usepackage{hyperref}

% Directory Structure
\graphicspath{{figures/}}

% Beamer configuration
\usetheme{Frankfurt}
%-----------------------------------------------------------------------------%
\AtBeginSection[]
{
   \begin{frame}
        \frametitle{Outline}
        \tableofcontents[currentsection]
    \end{frame}
}
%-----------------------------------------------------------------------------%
\title{Python for Scientists: Matplotlib}
%=============================================================================%
\begin{document}
% Listings settings
\lstset{
    language=Python,
    basicstyle=\ttfamily,
    showspaces=false,
    showstringspaces=false,
    backgroundcolor=\color{ltblue},
    numberstyle=\tiny\color{gray},        % line number style
    keywordstyle=\color{blue},          % keyword style
    commentstyle=\color{dkgreen},       % comment style
    stringstyle=\color{mauve}         % string literal style
}
%-----------------------------------------------------------------------------%
\begin{frame}
    \titlepage
\end{frame}
%=============================================================================%
\section[Intro]{What is Matplotlib?}
%-----------------------------------------------------------------------------%
\begin{frame}[fragile]
    \frametitle{What is Matplotlib?}
    
    \begin{itemize}
        \item Matplotlib is a rendering API for 2D (and limited 3D) plotting
        \item \lstinline|matplotlib.pyplot| is a streamlined Matlab-like interface to Matplotlib
        \item \lstinline|pylab| is a bundle of NumPy, SciPy and Matplotlib.\\
        One often sees it invoked like this:
        \begin{lstlisting}
from pylab import *
        \end{lstlisting}
    \end{itemize}
\end{frame}
%=============================================================================%
\section{2D Curve Plots}
%-----------------------------------------------------------------------------%
\frame{
    \frametitle{My First Plot}

    Plot $\sin x$ from $-2\pi$ to $2\pi$:
    \lstinputlisting{scripts/sine_plot.py}
}
%-----------------------------------------------------------------------------%
\begin{frame}[fragile]
    \frametitle{Saving your Figure}

    Append this to the end of your script:
    \begin{lstlisting}
plt.savefig('myplot.pdf')
    \end{lstlisting}
    It usually figures out the file type from the extension.
    \\~\\
    To remove space around the plot, use:
    \begin{lstlisting}
plt.savefig('myplot.pdf', bbox_inches='tight')
    \end{lstlisting}

\end{frame}
%-----------------------------------------------------------------------------%
\frame{
    \frametitle{Multiple Plots}

    Plot three sine curves with different phase shifts:
    $y_n = \sin(x + \phi_n)$
    \\~\\
    Call \lstinline|plot()| three times, then \lstinline|show()| the results
    \onslide<2->{
    \lstinputlisting{scripts/three_phase_plot.py}
    }
}
%-----------------------------------------------------------------------------%
\frame{
    \frametitle{Dots and Dashes}

    Stylise curves with dashes, shapes, and colours (like Matlab):

    \onslide<2->{
    \lstinputlisting{scripts/pretty_three_phase.py}
    }
}
%-----------------------------------------------------------------------------%
\frame{
    \frametitle{Axis Labels}

    Labeling axes is similar to Matlab:
    \lstinputlisting{scripts/labels.py}
}
%-----------------------------------------------------------------------------%
\frame{
    \frametitle{Legends}

    Include a legend in your plot
    \lstinputlisting{scripts/legend.py}
}
%=============================================================================%
\section{2D Field Plots}
%-----------------------------------------------------------------------------%
\frame{
    \frametitle{Contour Plots}

    Plot contours on range $[-2,2]\times[-2,2]$ for the function:
    \[ z(x, y) = \left(x - \frac{1}{2}\right) e^{-(x^2 + y^2)} \]

    \lstinputlisting{scripts/contour_plot.py}
}
%-----------------------------------------------------------------------------%
\begin{frame}[fragile]
    \frametitle{Image Plots}

    \lstinline|imshow| plots pixel fields (\lstinline|pcolor| is very slow)
    \lstinputlisting{scripts/image_plot.py}
\end{frame}
%=============================================================================%
\section{Object-Oriented Matplotlib}
%-----------------------------------------------------------------------------%
\begin{frame}[fragile]
    \frametitle{Object-Oriented Matplotlib}
    As your plots become more complex, you may need to start using object-oriented matplotlib
    \lstinputlisting{scripts/oo_plot.py}
\end{frame}
%-----------------------------------------------------------------------------%
\frame{
    \frametitle{Subplots}

    Put two plots on the same figure:
    \lstinputlisting{scripts/subplots.py}
}
%=============================================================================%
\section{Basemap}
%-----------------------------------------------------------------------------%
\frame{
    \frametitle{Basemap: Earth Grid Plotting}

    Basemap provides tools for plotting on several geographic grids.
}
%-----------------------------------------------------------------------------%
\frame{
    \frametitle{Orthographic Grids}

    \lstinputlisting{scripts/ortho_basemap.py}
}
%-----------------------------------------------------------------------------%
\frame{
    \frametitle{Mercator Grids}

    \lstinputlisting{scripts/merc_basemap.py}
}
%=============================================================================%
%-----------------------------------------------------------------------------%
%=============================================================================%
\end{document}
